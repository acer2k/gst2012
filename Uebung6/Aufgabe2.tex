\section*{Aufgabe 2}

Erfassung der Anforderungen
a) Leiten Sie aus der Spezifikation der Adressbuchsortierung aus Übung 5 alle Anforderungen an die Sortierung ab. Legen Sie eine Tabelle nach der Vorlage Anforderungen.pdf an und tragen Sie jede Anforderung mit einen aussagekräftigen Titel und Beschreibung ein.
b) Bewerten Sie die mit den Anforderungen verbundenen Funktionen bezüglich ihres Risikos für die geschäftliche Verwendung des Adressbuchs. Dokumentieren Sie das zugewiesene Risiko und priorisieren Sie die Anforderungen nach Wichtigkeit.
\subsection*{Anforderung an die Sortierung}

\begin{tabular}{rllll}
ID & Titel & Beschreibung & Risiko & Priorität\\
1 & Reihenfolge der Einträge & alphabetisch Sortiert nach DIN 5007, Variante 2, Ordnen von Schriftzeichenfolgen &  & \\
2 & Unterscheidungsmerkmale & nur Vor- und Nachname der Einträge &  & \\
3 & Priorität der Merkmale & Erst der Nachname, dann der Vornam &  & \\
4 & Groß- und Kleinschreibung & Die Groß- und Kleinschreibung der Wörter wird nicht berücksichtigt. (Alice = alicE) &  & \\
5 & Gleiche Einträge & Es können keine Einträge mit dem selben Vor- und Nachname eingetragen. &  & \\
6 & Vor- und Nachnamenzwang & Jeder Eintrag muss einen Vor- und einen Nachnamen haben. &  & \\
7 & Zulässige Zeichen im Namen & Name = [a-zA-Zäöüß]+ &  & \\
8 & Umschrift von Umlauten & ä = ae, ö = oe, ü = ue, ß = ss &  & \\
\end{tabular}
\subsection*{Risikobewertung}
